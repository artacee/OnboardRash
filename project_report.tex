\documentclass[11pt,a4paper]{article}

% Packages
\usepackage[utf8]{inputenc}
\usepackage[margin=0.9in]{geometry}
\usepackage{graphicx}
\usepackage{hyperref}
\usepackage{booktabs}
\usepackage{enumitem}
\usepackage{xcolor}
\usepackage{fancyhdr}
\usepackage{listings}
\usepackage{tabularx}

% Colors
\definecolor{codeblue}{RGB}{0, 102, 204}
\definecolor{codegray}{RGB}{100, 100, 100}
\definecolor{backgray}{RGB}{245, 245, 245}

% Code styling
\lstset{
    basicstyle=\ttfamily\small,
    backgroundcolor=\color{backgray},
    frame=single,
    framerule=0pt,
    breaklines=true
}

% Header/Footer
\pagestyle{fancy}
\fancyhf{}
\rhead{OnboardRash | Technical Overview}
\rfoot{\thepage}

\begin{document}

% Title
\begin{center}
    {\LARGE\textbf{OnboardRash}}\\[0.3cm]
    {\large Real-Time Rash Driving Detection System}\\[0.2cm]
    {\color{codegray} Technical Overview Document}\\[0.5cm]
    \hrule
\end{center}

\vspace{0.5cm}

%==============================================================================
\section*{Project Summary}
%==============================================================================

OnboardRash is an end-to-end IoT solution for detecting unsafe driving patterns in public transport. The system combines edge computing on Raspberry Pi with a cloud backend and real-time web dashboard.

\begin{tabularx}{\textwidth}{@{}lX@{}}
\textbf{Platform:} & Raspberry Pi 4 (Edge) + Flask Server (Cloud) \\
\textbf{Communication:} & REST API over 4G/LTE + WebSocket for real-time \\
\textbf{Database:} & SQLite (both edge \& cloud) \\
\textbf{Dashboard:} & HTML/CSS/JavaScript with Leaflet.js maps \\
\end{tabularx}

%==============================================================================
\section*{System Architecture}
%==============================================================================

\begin{lstlisting}
+-------------------+      HTTP/JSON      +------------------+
|   BUS UNIT (Pi)   | ------------------> |  FLASK BACKEND   |
|                   |                     |                  |
| - MPU6050 (IMU)   |      WebSocket      | - REST API       |
| - GPS Module      | <------------------ | - Socket.IO      |
| - Camera          |                     | - SQLite DB      |
| - Ultrasonic      |                     +--------+---------+
| - Kalman Filter   |                              |
| - Local SQLite    |                              v
+-------------------+                     +------------------+
                                          |  WEB DASHBOARD   |
                                          | - Live Map       |
                                          | - Alert Panel    |
                                          | - Events Table   |
                                          +------------------+
\end{lstlisting}

%==============================================================================
\section*{Hardware Stack}
%==============================================================================

\begin{tabular}{@{}llp{7cm}@{}}
\toprule
\textbf{Component} & \textbf{Model} & \textbf{Function} \\
\midrule
Microcontroller & Raspberry Pi 4 & Main processing unit \\
IMU & MPU-6050 & 3-axis accelerometer (X/Y/Z in G-force) \\
GPS & NEO-6M & Location \& speed tracking \\
Camera & Pi Camera v3 & Front-facing, 640x480 @ 15fps \\
Distance Sensor & HC-SR04 & Ultrasonic (left-side clearance) \\
Connectivity & 4G LTE HAT & Cellular data transmission \\
\bottomrule
\end{tabular}

%==============================================================================
\section*{Detection Logic}
%==============================================================================

Events are triggered based on sensor thresholds:

\begin{tabular}{@{}lll@{}}
\toprule
\textbf{Event} & \textbf{Trigger} & \textbf{Severity} \\
\midrule
HARSH\_BRAKE & Accel X < -1.5g & HIGH if X < -1.8g \\
HARSH\_ACCEL & Accel X > 1.0g & MEDIUM \\
AGGRESSIVE\_TURN & Accel Y > 0.8g & HIGH if Y > 1.0g \\
TAILGATING & Vehicle in frame > 15\% & MEDIUM \\
CLOSE\_OVERTAKING & Distance < 100cm, duration > 0.5s & MEDIUM \\
\bottomrule
\end{tabular}

\vspace{0.3cm}
\textbf{Sensor Fusion:} A Kalman Filter combines GPS speed with IMU acceleration for accurate velocity estimation, handling GPS dropouts and vibration noise.

%==============================================================================
\section*{API Endpoints}
%==============================================================================

\begin{tabular}{@{}llp{6cm}@{}}
\toprule
\textbf{Method} & \textbf{Endpoint} & \textbf{Description} \\
\midrule
POST & /api/events & Receive event from bus unit \\
GET & /api/events & List events (with filters) \\
GET & /api/stats & Dashboard summary stats \\
GET & /api/buses & List registered buses \\
POST & /api/buses/\{id\}/location & Update bus GPS position \\
GET & /api/buses/locations & All active bus positions \\
GET & /api/export/events & CSV download \\
\bottomrule
\end{tabular}

\vspace{0.3cm}
\textbf{Authentication:} API endpoints protected with \texttt{X-API-Key} header.

%==============================================================================
\section*{Data Flow}
%==============================================================================

\begin{enumerate}[noitemsep]
    \item \textbf{Sensor Read:} Pi reads IMU, GPS, Camera, Ultrasonic at 10Hz
    \item \textbf{Fusion:} Kalman filter fuses GPS speed with IMU acceleration
    \item \textbf{Detection:} Logic engine checks thresholds for each event type
    \item \textbf{Evidence:} If event detected, save 10s video clip + snapshot
    \item \textbf{Queue:} Event JSON + media paths stored in local SQLite
    \item \textbf{Sync:} Background thread uploads queued events via HTTP POST
    \item \textbf{Broadcast:} Backend emits \texttt{new\_alert} via Socket.IO
    \item \textbf{Dashboard:} JS client receives WebSocket event, updates UI
\end{enumerate}

%==============================================================================
\section*{Offline Handling}
%==============================================================================

When 4G connectivity is lost:
\begin{itemize}[noitemsep]
    \item Events are stored in local SQLite queue (\texttt{events\_queue.db})
    \item Background sync thread retries every 5 seconds
    \item On reconnect, all queued events are uploaded in order
    \item Dashboard receives backlogged events with original timestamps
\end{itemize}

%==============================================================================
\section*{Dashboard Features}
%==============================================================================

\begin{tabular}{@{}lp{9cm}@{}}
\toprule
\textbf{Feature} & \textbf{Implementation} \\
\midrule
Live Map & Leaflet.js with CARTO dark tiles, bus markers refresh every 10s \\
Real-time Alerts & Socket.IO client, audio beep for HIGH severity \\
Events Table & Filterable by severity/type, paginated \\
Stats Cards & Today's events, active buses, high severity count \\
CSV Export & Server-side generation via /api/export \\
Evidence Viewer & Modal popup for snapshots, video playback \\
\bottomrule
\end{tabular}

%==============================================================================
\section*{Tech Stack Summary}
%==============================================================================

\begin{tabular}{@{}ll@{}}
\toprule
\textbf{Layer} & \textbf{Technologies} \\
\midrule
Edge (Pi) & Python 3, smbus2, pyserial, OpenCV, picamera2 \\
Backend & Flask, Flask-SocketIO, SQLAlchemy, Flask-CORS \\
Database & SQLite (file-based, both edge and cloud) \\
Frontend & HTML5, CSS3, Vanilla JS, Socket.IO client, Leaflet.js \\
Protocol & REST (JSON), WebSocket (Socket.IO) \\
\bottomrule
\end{tabular}

%==============================================================================
\section*{Metrics \& Testing}
%==============================================================================

\begin{itemize}[noitemsep]
    \item Sensor sample rate: 10 Hz (100ms loop)
    \item Event transmission latency: < 500ms (online)
    \item Dashboard update rate: 10s (map), instant (alerts via WebSocket)
    \item Simulator: 3 virtual buses generating events every 2s for testing
\end{itemize}

%==============================================================================
\section*{Deployment}
%==============================================================================

\begin{lstlisting}
# Start backend
cd backend && python app.py

# Start simulator (for testing without hardware)
python simulator/simulator.py

# OR run full demo (backend + dashboard + simulator)
python demo.py
\end{lstlisting}

\vspace{0.5cm}
\hrule
\vspace{0.3cm}
{\small\color{codegray} Prepared for internship project review | January 2026}

\end{document}
